\documentclass[conference]{IEEEtran}
\IEEEoverridecommandlockouts
% The preceding line is only needed to identify funding in the first footnote. If that is unneeded, please comment it out.
\usepackage{cite}
\usepackage{amsmath,amssymb,amsfonts}
\usepackage{algorithmic}
\usepackage{graphicx}
\usepackage{hyperref}
\usepackage{textcomp}
\usepackage{xcolor}
\def\BibTeX{{\rm B\kern-.05em{\sc i\kern-.025em b}\kern-.08em
    T\kern-.1667em\lower.7ex\hbox{E}\kern-.125emX}}
\begin{document}

\title{Data Science and Machine Learning Project}

\author{\IEEEauthorblockN{Md Jahidul Islam}
\IEEEauthorblockA{\textit{Computer and Information Sciences} \\
\textit{Georgia Southern University}\\
Georgia, USA \\
mi03289@georgiasouthern.edu}
}

\maketitle

\begin{abstract}
An abstract section should provide a concise summary of the entire research work. It briefly introduces the research problem and explains its importance. The abstract states the main objective or goal of the study. It summarizes the methodology or approach used to conduct the research. It highlights the key results or findings. Finally, it indicates the significance or contribution of the study in a clear and compact manner.

\end{abstract}

Integration of Latex and GitHub Link: \href{https://github.com/JahidulGUB-Islam/LaTeX-Article-and-Linking-to-GitHub-DSML}{GitHub Repository}

\begin{IEEEkeywords}
Component, formatting, style, styling, insert and etc.
\end{IEEEkeywords}

\section{Introduction}
A research introduction should clearly present the context and purpose of the study. It begins by introducing the background of the topic and explaining why the problem is important. The introduction then highlights gaps or limitations in existing research. It states the main objective or research question of the study. Finally, it briefly indicates the approach used and the significance of the research outcomes.

\section{Related Works}
The related works section reviews existing studies relevant to the research topic and summarizes their key contributions. It compares different methods, models, or approaches used by previous researchers. This section identifies strengths and limitations of existing work to highlight research gaps. It shows how the current study differs from or improves upon earlier research. Finally, it establishes the foundation and motivation for the proposed approach.

\section{Proposed Methodology}

The methodology section explains how the research was conducted in a clear and detailed manner. It describes the research design, techniques, and tools used to collect and analyze data. It explains the procedures, experiments, or algorithms applied in the study. The section justifies why these methods were chosen and how they help achieve the research objectives. It also includes any assumptions, parameters, or constraints considered. Overall, it ensures that the study can be understood, evaluated, and replicated by others.


\section{Result anlysis and Discussion}
The Result Analysis section presents and interprets the findings of the research in a clear and organized manner. It includes tables, charts, or graphs to illustrate key results. The section explains patterns, trends, or relationships observed in the data. It compares the findings with expected outcomes or previous studies to highlight similarities or differences. Any significant observations, anomalies, or insights are discussed. Overall, this section demonstrates how the results support the research objectives and contribute to the study’s conclusions.

\section{Conclusion}
The Conclusion section summarizes the main findings of the research and reflects on their significance. It restates the research objectives and explains how they were achieved. The section highlights the contributions of the study and its implications for the field. It may also discuss limitations encountered during the research. Finally, it suggests possible directions for future work or improvements. Overall, the conclusion provides a clear and concise closure to the study.


\begin{thebibliography}{00}
\bibitem{b1} G. Eason, B. Noble, and I. N. Sneddon, ``On certain integrals of Lipschitz-Hankel type involving products of Bessel functions,'' Phil. Trans. Roy. Soc. London, vol. A247, pp. 529--551, April 1955.
\bibitem{b2} J. Clerk Maxwell, A Treatise on Electricity and Magnetism, 3rd ed., vol. 2. Oxford: Clarendon, 1892, pp.68--73.
\bibitem{b3} I. S. Jacobs and C. P. Bean, ``Fine particles, thin films and exchange anisotropy,'' in Magnetism, vol. III, G. T. Rado and H. Suhl, Eds. New York: Academic, 1963, pp. 271--350.
\end{thebibliography}
\vspace{12pt}
\color{red}


\end{document}
